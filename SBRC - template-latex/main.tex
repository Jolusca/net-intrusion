\documentclass[12pt]{article}

% -------------------------
% Template SBRC / SBC
% -------------------------
\usepackage{sbc-template}
\usepackage{float}
\usepackage{graphicx}
\usepackage{url}
\usepackage{amsmath}
\usepackage{amsfonts}
\usepackage{amssymb}

\sloppy

\title{Estudo da Influência das Features e do Desempenho dos Classificadores na Detecção de Anomalias em Redes}

\author{
João Lucas Oliveira Mota\inst{1},
João Lucas Rodrigues da Silva\inst{2},
José Alberto Rodrigues Neto\inst{2}
}

\address{
Departamento de Engenharia de Teleinformática -- Universidade Federal do Ceará (UFC)\\
Fortaleza -- CE -- Brasil
\nextinstitute
Departamento de Computação -- Universidade Federal do Ceará (UFC)\\
Fortaleza -- CE -- Brasil
\email{\{jlucasoliveira2002\}@gmail.com,
\{joao.lsilva,albertoo11\}@alu.ufc.br}
}

\begin{document}

\maketitle

\begin{resumo}
Este trabalho investiga o impacto da seleção e remoção manual de atributos no desempenho de classificadores de Aprendizado de Máquina para detecção de anomalias em redes de computadores. Utilizando o conjunto de dados Network Intrusion Detection, disponibilizado publicamente por Sampada Bhosale \cite{dataset_kaggle}, foram avaliados os modelos Random Forest (RF), K-Nearest Neighbors (KNN), Multilayer Perceptron (MLP) e Isolation Forest (IF). Os experimentos simularam cenários de indisponibilidade de dados através da exclusão controlada de features críticas, como o parâmetro src\_bytes. Os resultados demonstraram que o Random Forest obteve o melhor desempenho inicial, porém apresentou maior sensibilidade à perda de informações, com queda no desempenho ao remover atributos influentes. Em contraste, o MLP revelou maior robustez e capacidade de generalização, mantendo F1-Scores superiores a 0,99 mesmo com conjuntos reduzidos. O estudo conclui que a escolha do classificador deve considerar não apenas a precisão absoluta, mas a resiliência do modelo em ambientes reais onde a integridade dos dados pode ser comprometida.
\end{resumo}

% -------------------------
% Seções
% -------------------------
\section{Introdução}

O crescimento contínuo da demanda por serviços digitais, tanto no cotidiano quanto em ambientes corporativos, tem impulsionado a ampla utilização de redes de computadores, incluindo data centers e servidores privados virtuais (VPS). Como consequência, observou-se um aumento significativo no número e na complexidade das ameaças, de acordo com o Relatório de Índice de Cibersegurança da ENISA (2024)\cite{enisa2025}, embora haja um esforço para fortalecer a resiliência cibernética, a capacidade do setor privado em detectar e analisar incidentes ainda enfrenta desafios de maturidade técnica. Essa evolução torna a detecção de intrusões um dos principais desafios da área de segurança da informação. Ataques cada vez mais sofisticados exigem mecanismos capazes de identificar comportamentos anômalos de forma precisa e eficiente, mesmo em cenários caracterizados por grandes volumes de dados e alta dimensionalidade.

Nesse contexto, técnicas de aprendizado de máquina têm sido amplamente empregadas em sistemas de detecção de intrusões\cite{buczak2015survey}\cite{zamani2013machine}\cite{he2023adversarial}, devido à sua capacidade de aprender padrões complexos a partir do tráfego de rede e de se adaptar a diferentes tipos de ataques. Historicamente, a eficácia desses sistemas depende da qualidade das bases de dados utilizadas para o treinamento. Conforme analisado previamente no artigo "A detailed analysis of the KDD CUP 99 data set" \cite{tavallaee2009detailed}, cojunto de dados podem apresentar deficiências como a redundância de registros que foi avaliado no conjunto do KDD CUP 99, dataset criado para uma competição da área de mineração de dados, baseando-se em tráfego de rede simulado para distinguir entre conexões normais e invasões. O que pode causar um viés nos classificadores em direção a padrões frequentes, resultando em taxas de acerto inflacionadas que não refletem o desempenho real em redes modernas.

Diversos estudos na literatura investigam estratégias automáticas de redução de dimensionalidade e seleção de atributos, como a Análise de Componentes Principais (PCA), com o objetivo de melhorar a eficiência computacional e a capacidade de generalização dos modelos\cite{jia2022feature}\cite{mladenic2005feature}\cite{zaheer2025evaluating}. Apesar disso, ainda há uma lacuna na compreensão do impacto individual das features mais relevantes no desempenho de diferentes classificadores, especialmente quando tais atributos deixam de estar disponíveis. Dessa forma, este trabalho tem como objetivo analisar a influência da seleção e da remoção de atributos no desempenho de diferentes modelos de aprendizado de máquina aplicados à detecção de anomalias em redes de computadores. Avaliando modelos supervisionados e não supervisionados, incluindo Random Forest, K-Nearest Neighbors (KNN), Multilayer Perceptron (MLP) e Isolation Forest. Os experimentos buscam, assim, fornecer uma análise comparativa que contribua para a compreensão da robustez e das limitações de cada abordagem em cenários realistas de detecção de intrusões.

\section{Dataset}

O conjunto de dados utilizado foi obtido a partir da plataforma Kaggle, no repositório
\textit{Network Intrusion Detection}, disponibilizado publicamente por
Sampada Bhosale~\cite{dataset_kaggle}.

O dataset foi construído a partir da simulação de um ambiente de rede militar típico da
Força Aérea dos Estados Unidos, desenvolvido para representar diferentes tipos de tráfego
de rede, incluindo tanto comunicações legítimas quanto diversos tipos de ataques.
Cada instância do conjunto de dados representa uma conexão de rede, contendo atributos
como a quantidade de bytes transmitidos pela origem, a duração da conexão e outros
parâmetros, conforme descrito na tabela apresentada no Apêndice.

O conjunto de dados contém aproximadamente 25 mil instâncias, descritas por 42
atributos. Após a simulação supervisionada do ambiente, as conexões foram rotuladas como
tráfego normal ou intrusão, caracterizando um problema de classificação binária, no qual
a variável alvo assume os valores \textit{Normal} ou \textit{Anomalia}. A descrição
detalhada dos atributos utilizados é apresentada no Apêndice deste trabalho.
\section{Pré-processamento}
\subsection{Tratamento de Dados}
Devido às diferenças entre os modelos avaliados, as etapas de pré-processamento
foram adaptadas aos requisitos específicos de cada classificador, mantendo a
comparabilidade dos resultados. Inicialmente, o conjunto de dados foi separado
em atributos preditores e rótulos, seguido da divisão em conjuntos de treinamento
e teste por meio de amostragem estratificada.

As variáveis categóricas foram codificadas por meio de \textit{One-Hot Encoding}
ou \textit{Label Encoding}, conforme o modelo utilizado, enquanto as variáveis
numéricas foram padronizadas quando necessário. Além disso, foram conduzidos
experimentos com conjuntos de dados reduzidos, nos quais atributos relevantes
foram removidos de forma controlada, com o objetivo de avaliar a robustez dos
modelos frente à ausência parcial de informações.


\subsection{Seleção e remoção de Atributos}
A aplicação da Análise de Componentes Principais (PCA) como etapa de pré-processamento
tem se mostrado eficaz para a redução da dimensionalidade dos dados sem perda
significativa de informação. Estudos indicam que o PCA é capaz de preservar cerca de
99\% da variância original mesmo com reduções superiores a 50\% no número de atributos,
conforme observado por Santos e Miani~\cite{santos_reducao_dimensao_intrusao}. Nesse
sentido, técnicas de redução e seleção de atributos são amplamente empregadas na
literatura como forma de mitigar redundâncias e melhorar a generalização dos modelos.

Neste artigo, entretanto, optou-se por não aplicar técnicas automáticas de redução de
dimensionalidade, como o PCA. Em vez disso, foram realizadas diferentes análises
utilizando modelos de aprendizagem de máquina com o objetivo de identificar as
\textit{features} de maior importância para cada classificador. Em seguida, buscando
simular um cenário mais próximo de um ambiente real, no qual nem sempre todos os
atributos estão disponíveis para os algoritmos de análise, foram conduzidos testes
com a remoção manual das \textit{features} mais relevantes. Essa abordagem permitiu
analisar a influência individual de cada atributo no desempenho dos modelos, bem como
avaliar a robustez de cada método frente à ausência parcial de informações.





\section{Metodologia}

\subsection{Modelos Utilizados}

\subsubsection{Random Forest}
O Random Forest (RF) é um algoritmo de aprendizado de máquina supervisionado baseado em
conjuntos de árvores de decisão, no qual múltiplas árvores são treinadas a partir de
subconjuntos aleatórios dos dados e dos atributos, combinando suas previsões por meio de
votação majoritária. Sua utilização neste trabalho se deve à robustez frente a dados de
alta dimensionalidade, à capacidade de modelar relações não lineares e à possibilidade de
analisar a importância das features, características relevantes para problemas de
detecção de intrusão em redes~\cite{breiman_random_forest}.

\subsubsection{K-Nearest Neighbors}

Outro modelo utilizado foi o K-Nearest Neighbors (KNN). Esse modelo se diferencia dos
demais por não realizar uma etapa explícita de aprendizado a partir dos dados de
treinamento. Em vez disso, ele armazena as instâncias conhecidas e classifica uma nova
entrada com base na similaridade em relação a um número arbitrário ($K$) de exemplos
mais próximos previamente registrados~\cite{cover_hart_knn}

\subsubsection{Isolation Forest}

O Isolation Forest (IF) é um algoritmo de aprendizado de máquina não supervisionado,
projetado especificamente para a detecção de anomalias (ou outliers) em conjuntos de
dados. A escolha do Isolation Forest foi motivada pelo fato de que, em alguns cenários,
não é possível obter todos os dados rotulados necessários para a detecção de intrusões
em redes. Dessa forma, buscou-se avaliar se o IF seria capaz de identificar anomalias
mesmo na ausência parcial dos rótulos utilizados no treinamento~\cite{liu_isolation_forest}.

\subsubsection{Multilayer Perceptron (MLP)}
O Multilayer Perceptron (MLP) é um modelo de rede neural artificial supervisionado,
composto por camadas de neurônios interconectados que aplicam transformações não lineares
sobre os dados de entrada. A escolha do MLP foi motivada por sua capacidade de aprender
padrões complexos no tráfego de rede, sendo adequado para a detecção de intrusões que
apresentam comportamentos sutis. 

Neste trabalho, o modelo foi avaliado em dois cenários:
um utilizando todas as features disponíveis e outro com a remoção da feature mais
influente identificada no modelo de melhor resultado, com o objetivo de analisar o impacto da
redução de atributos em seu desempenho. 

Ressalta-se que o desempenho do MLP depende
fortemente da escolha adequada de hiperparâmetros e do pré-processamento dos dados,
conforme discutido na literatura~\cite{haykin_neural_networks}.


\section{Resultados e Discussão}

\subsection{Resultados com K-Nearest Neighbors}

Após alguns testes, os resultados mais precisos foram encontrados com $K=1$, resultando
em um F1-Score de 0.9946. Provavelmente, esse comportamento ocorre devido à grande
quantidade de dados disponíveis para cada entrada de treinamento, o que torna a maioria
dos ataques mais evidentes. Valores superiores de $K$ levaram ao sobreajuste e ao
decréscimo progressivo do desempenho do modelo.

Após uma análise baseada na remoção individual de cada atributo, observou-se que o
parâmetro \textit{hot} foi o mais decisivo para o modelo, reduzindo o F1-Score para
0.9931 quando removido. Outra informação relevante é que oito atributos, quando
removidos, aumentaram a precisão do modelo. Após essa remoção, o F1-Score atingiu
0.9960, sendo o atributo \textit{diff\_srv\_rate} o mais prejudicial, cuja exclusão
resultou em um aumento de 0.0005 pontos no F1-Score.

\begin{table}[ht]
\centering
\caption{Métricas de desempenho do KNN}
\label{tab:if_metricsKNN}
\begin{tabular}{lcc}
\hline
\textbf{Métrica} & \textbf{Classe Normal} & \textbf{Classe Anomalia} \\
\hline
Precision & 0.99 & 0.99 \\
Recall    & 0.99 & 0.99 \\
F1-Score  & 0.99 & 0.99 \\
Support   & 4034 & 3524 \\
\hline
\end{tabular}
\end{table}

\subsection{Resultados com Isolation Forest}

No entanto, o Isolation Forest se mostrou ineficiente quando comparado aos outros
modelos utilizados durante a pesquisa, apresentando a menor acurácia (0.748).
Entretanto, a acurácia não é a métrica mais confiável em conjuntos de dados
desbalanceados, o que é comum em problemas de detecção de anomalias, pois pode ser
inflacionada pela capacidade do modelo em classificar corretamente a classe
majoritária (tráfego normal). Dessa forma, uma análise mais aprofundada torna-se
necessária.

\begin{table}[ht]
\centering
\caption{Métricas de desempenho do Isolation Forest}
\label{tab:if_metricsIF}
\begin{tabular}{lcc}
\hline
\textbf{Métrica} & \textbf{Classe Normal} & \textbf{Classe Anomalia} \\
\hline
Precision & 1.00 & 0.65 \\
Recall    & 0.53 & 1.00 \\
F1-Score  & 0.69 & 0.79 \\
Support   & 2690 & 2349 \\
\hline
\end{tabular}
\end{table}

Como pode ser observado na Tabela~\ref{tab:if_metrics}, todos os eventos classificados
como normais eram, de fato, normais. A principal dificuldade do modelo esteve na
identificação correta das anomalias, apresentando precisão de 65\% (0.65) ao rotular
eventos como intrusão. Além disso, considerando todos os eventos analisados, o modelo
foi capaz de identificar corretamente apenas 53\% (0.53) dos eventos normais, o que
indica a geração de um elevado número de falsos positivos, isto é, eventos normais
classificados como ataques.

Por outro lado, o Isolation Forest não apresentou o problema de gerar falsos negativos,
uma vez que ataques não foram classificados como eventos normais. Em síntese, o modelo
apresenta dificuldades em verificar se um evento é realmente normal, gerando falsos
positivos que podem ocasionar retrabalho para equipes de segurança, as quais precisam
analisar manualmente eventos normais identificados como anômalos.

\subsection{Resultados com Random Forest}
CHANGE ME
% Texto a ser inserido

\subsection{Resultados com MLP}
CHANGE ME
% Texto a ser inserido



\section{Conclusão}

CHANGE ME

Este trabalho analisou a influência da seleção de atributos e do desempenho de
diferentes classificadores na detecção de anomalias em redes de computadores.


\bibliographystyle{sbc}
\bibliography{referencias}
\clearpage
\appendix
\section*{Apêndice A — Descrição Completa do Dataset}

\begin{table}[H]
\centering
\caption{Descrição completa dos atributos do dataset de intrusão em redes}
\label{tab:dataset_full}
\footnotesize
\begin{tabular}{|p{4cm}|p{11cm}|}
\hline
\textbf{Atributo} & \textbf{Descrição} \\ \hline
duration & Duração da conexão de rede \\ \hline
protocol\_type & Protocolo utilizado na conexão (TCP, UDP ou ICMP) \\ \hline
service & Serviço de rede acessado \\ \hline
flag & Estado da conexão conforme o protocolo \\ \hline
src\_bytes & Quantidade de bytes enviados pela origem \\ \hline
dst\_bytes & Quantidade de bytes enviados pelo destino \\ \hline
land & Indica se origem e destino possuem mesmo IP e porta \\ \hline
wrong\_fragment & Número de fragmentos incorretos \\ \hline
urgent & Número de pacotes marcados como urgentes \\ \hline
hot & Indicadores de comportamentos suspeitos \\ \hline
num\_failed\_logins & Número de tentativas de login malsucedidas \\ \hline
logged\_in & Indica se o login foi realizado com sucesso \\ \hline
num\_compromised & Número de condições comprometidas \\ \hline
root\_shell & Indica obtenção de shell com privilégio root \\ \hline
su\_attempted & Tentativas de uso do comando \textit{su} \\ \hline
num\_root & Número de acessos root \\ \hline
num\_file\_creations & Número de arquivos criados \\ \hline
num\_shells & Número de shells abertos \\ \hline
num\_access\_files & Número de acessos a arquivos sensíveis \\ \hline
num\_outbound\_cmds & Número de comandos externos enviados \\ \hline
is\_host\_login & Indica login como host \\ \hline
is\_guest\_login & Indica login como convidado \\ \hline
count & Conexões com o mesmo host em janela de tempo \\ \hline
srv\_count & Conexões com o mesmo serviço em janela de tempo \\ \hline
serror\_rate & Taxa de erros SYN \\ \hline
srv\_serror\_rate & Taxa de erros SYN para o serviço \\ \hline
rerror\_rate & Taxa de erros de resposta \\ \hline
srv\_rerror\_rate & Taxa de erros de resposta para o serviço \\ \hline
same\_srv\_rate & Taxa de conexões para o mesmo serviço \\ \hline
diff\_srv\_rate & Taxa de conexões para serviços diferentes \\ \hline
srv\_diff\_host\_rate & Taxa de serviços acessando hosts distintos \\ \hline
dst\_host\_count & Número de conexões para o mesmo host destino \\ \hline
dst\_host\_srv\_count & Número de conexões para o mesmo serviço no host \\ \hline
dst\_host\_same\_srv\_rate & Taxa de serviços iguais para o host destino \\ \hline
dst\_host\_diff\_srv\_rate & Taxa de serviços diferentes para o host destino \\ \hline
dst\_host\_same\_src\_port\_rate & Taxa de conexões com mesma porta de origem \\ \hline
dst\_host\_srv\_diff\_host\_rate & Taxa de serviços com hosts distintos \\ \hline
dst\_host\_serror\_rate & Taxa de erros SYN no host destino \\ \hline
dst\_host\_srv\_serror\_rate & Taxa de erros SYN por serviço no host destino \\ \hline
dst\_host\_rerror\_rate & Taxa de erros de resposta no host destino \\ \hline
dst\_host\_srv\_rerror\_rate & Taxa de erros de resposta por serviço no host destino \\ \hline
\end{tabular}
\end{table}


\end{document}
