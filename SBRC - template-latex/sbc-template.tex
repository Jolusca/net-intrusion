\documentclass[12pt]{article}

% -------------------------
% Template SBRC / SBC
% -------------------------
\usepackage{sbc-template}
\usepackage{float}
\usepackage{graphicx}
\usepackage{url}
\usepackage{amsmath}
\usepackage{amsfonts}
\usepackage{amssymb}

\sloppy

\title{Estudo da Influência das Features e do Desempenho dos Classificadores na Detecção de Anomalias em Redes}

\author{
João Lucas Oliveira Mota\inst{1},
João Lucas Rodrigues da Silva\inst{2},
José Alberto Rodrigues Neto\inst{2}
}

\address{
Departamento de Engenharia de Teleinformática -- Universidade Federal do Ceará (UFC)\\
Fortaleza -- CE -- Brasil
\nextinstitute
Departamento de Computação -- Universidade Federal do Ceará (UFC)\\
Fortaleza -- CE -- Brasil
\email{\{jlucasoliveira2002\}@gmail.com,
\{joaolucas.rodrigues,jose.alberto\}@alu.ufc.br}
}

\begin{document}

\maketitle

\begin{abstract}
This document is a model and instructions for \LaTeX.
This and the sbc-template style define the components of your paper.
Do not use symbols, special characters, footnotes, or math in the paper title or abstract.
\end{abstract}

\begin{resumo}
Este documento apresenta o modelo e o conjunto de instruções para artigos elaborados em \LaTeX\ segundo o padrão da SBC.
O arquivo \textit{sbc-template} define os principais componentes do artigo, como título, autores e seções.
\end{resumo}

\section{Introdução}
CHANGE ME
Este artigo investiga a influência da seleção de atributos e do desempenho de diferentes
classificadores na detecção de anomalias em tráfego de redes de computadores.



\section{Dataset e pré-processamento}
\subsection{Dataset}

O conjunto de dados utilizado foi obtido a partir da plataforma Kaggle, no repositório
\textit{Network Intrusion Detection}, disponibilizado publicamente por
Sampada Bhosale~\cite{dataset_kaggle}.


O dataset foi construído a partir da simulação de um ambiente de rede militar típico da Força Aérea dos Estados Unidos, desenvolvido
para representar diferentes tipos de tráfego de rede, incluindo tanto comunicações legítimas quanto diversos tipos de ataques.
Cada instância do conjunto de dados representa uma conexão de rede, contendo atributos como a quantidade de bytes transmitidos pela origem,
a duração da conexão e outros parâmetros, conforme descrito na tabela apresentada no Apêndice.

Por fim, após as simulações realizadas de forma supervisionada, as conexões foram classificadas como tráfego normal ou como intrusão, 
fazendo com que a variável alvo seja do tipo binária, assumindo os valores Normal ou Anomalia.

\subsection{Pré-processamento}
CHANGE ME A aplicação de PCA como etapa de pré-processamento tem se mostrado eficaz para reduzir a dimensionalidade 
dos dados sem perda significativa de informação, preservando cerca de 99\% da variância original mesmo 
com reduções superiores a 50\% no número de atributos, conforme observado por Santos e Miani~\cite{santos_reducao_dimensao_intrusao}.

\section*{Agradecimentos}

Os autores agradecem à Universidade Federal do Ceará (UFC) pelo suporte institucional.



\bibliographystyle{sbc}
\bibliography{referencias}

\section*{Apêndice A — Descrição Completa do Dataset}
\begin{table}[H]
\centering
\caption{Descrição completa dos atributos do dataset de intrusão em redes}
\label{tab:dataset_full}
\footnotesize
\begin{tabular}{|p{4cm}|p{11cm}|}
\hline
\textbf{Atributo} & \textbf{Descrição} \\ \hline
duration & Duração da conexão de rede \\ \hline
protocol\_type & Protocolo utilizado na conexão (TCP, UDP ou ICMP) \\ \hline
service & Serviço de rede acessado \\ \hline
flag & Estado da conexão conforme o protocolo \\ \hline
src\_bytes & Quantidade de bytes enviados pela origem \\ \hline
dst\_bytes & Quantidade de bytes enviados pelo destino \\ \hline
land & Indica se origem e destino possuem mesmo IP e porta \\ \hline
wrong\_fragment & Número de fragmentos incorretos \\ \hline
urgent & Número de pacotes marcados como urgentes \\ \hline
hot & Indicadores de comportamentos suspeitos \\ \hline
num\_failed\_logins & Número de tentativas de login malsucedidas \\ \hline
logged\_in & Indica se o login foi realizado com sucesso \\ \hline
num\_compromised & Número de condições comprometidas \\ \hline
root\_shell & Indica obtenção de shell com privilégio root \\ \hline
su\_attempted & Tentativas de uso do comando \textit{su} \\ \hline
num\_root & Número de acessos root \\ \hline
num\_file\_creations & Número de arquivos criados \\ \hline
num\_shells & Número de shells abertos \\ \hline
num\_access\_files & Número de acessos a arquivos sensíveis \\ \hline
num\_outbound\_cmds & Número de comandos externos enviados \\ \hline
is\_host\_login & Indica login como host \\ \hline
is\_guest\_login & Indica login como convidado \\ \hline
count & Conexões com o mesmo host em janela de tempo \\ \hline
srv\_count & Conexões com o mesmo serviço em janela de tempo \\ \hline
serror\_rate & Taxa de erros SYN \\ \hline
srv\_serror\_rate & Taxa de erros SYN para o serviço \\ \hline
rerror\_rate & Taxa de erros de resposta \\ \hline
srv\_rerror\_rate & Taxa de erros de resposta para o serviço \\ \hline
same\_srv\_rate & Taxa de conexões para o mesmo serviço \\ \hline
diff\_srv\_rate & Taxa de conexões para serviços diferentes \\ \hline
srv\_diff\_host\_rate & Taxa de serviços acessando hosts distintos \\ \hline
dst\_host\_count & Número de conexões para o mesmo host destino \\ \hline
dst\_host\_srv\_count & Número de conexões para o mesmo serviço no host \\ \hline
dst\_host\_same\_srv\_rate & Taxa de serviços iguais para o host destino \\ \hline
dst\_host\_diff\_srv\_rate & Taxa de serviços diferentes para o host destino \\ \hline
dst\_host\_same\_src\_port\_rate & Taxa de conexões com mesma porta de origem \\ \hline
dst\_host\_srv\_diff\_host\_rate & Taxa de serviços com hosts distintos \\ \hline
dst\_host\_serror\_rate & Taxa de erros SYN no host destino \\ \hline
dst\_host\_srv\_serror\_rate & Taxa de erros SYN por serviço no host destino \\ \hline
dst\_host\_rerror\_rate & Taxa de erros de resposta no host destino \\ \hline
dst\_host\_srv\_rerror\_rate & Taxa de erros de resposta por serviço no host destino \\ \hline
\end{tabular}
\end{table}

\end{document}
