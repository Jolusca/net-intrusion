\section{Metodologia}

\subsection{Modelos Utilizados}

\subsubsection{Random Forest}
O Random Forest (RF) é um algoritmo de aprendizado de máquina supervisionado baseado em
conjuntos de árvores de decisão, no qual múltiplas árvores são treinadas a partir de
subconjuntos aleatórios dos dados e dos atributos, combinando suas previsões por meio de
votação majoritária. Sua utilização neste trabalho se deve à robustez frente a dados de
alta dimensionalidade, à capacidade de modelar relações não lineares e à possibilidade de
analisar a importância das features, características relevantes para problemas de
detecção de intrusão em redes~\cite{breiman_random_forest}.

\subsubsection{K-Nearest Neighbors}

Outro modelo utilizado foi o K-Nearest Neighbors (KNN). Esse modelo se diferencia dos
demais por não realizar uma etapa explícita de aprendizado a partir dos dados de
treinamento. Em vez disso, ele armazena as instâncias conhecidas e classifica uma nova
entrada com base na similaridade em relação a um número arbitrário ($K$) de exemplos
mais próximos previamente registrados~\cite{cover_hart_knn}

\subsubsection{Isolation Forest}

O Isolation Forest (IF) é um algoritmo de aprendizado de máquina não supervisionado,
projetado especificamente para a detecção de anomalias (ou outliers) em conjuntos de
dados. A escolha do Isolation Forest foi motivada pelo fato de que, em alguns cenários,
não é possível obter todos os dados rotulados necessários para a detecção de intrusões
em redes. Dessa forma, buscou-se avaliar se o IF seria capaz de identificar anomalias
mesmo na ausência parcial dos rótulos utilizados no treinamento~\cite{liu_isolation_forest}.

\subsubsection{Multilayer Perceptron (MLP)}
O Multilayer Perceptron (MLP) é um modelo de rede neural artificial supervisionado,
composto por camadas de neurônios interconectados que aplicam transformações não lineares
sobre os dados de entrada. A escolha do MLP foi motivada por sua capacidade de aprender
padrões complexos no tráfego de rede, sendo adequado para a detecção de intrusões que
apresentam comportamentos sutis. 

Neste trabalho, o modelo foi avaliado em dois cenários:
um utilizando todas as features disponíveis e outro com a remoção da feature mais
influente identificada no modelo de melhor resultado, com o objetivo de analisar o impacto da
redução de atributos em seu desempenho. 

Ressalta-se que o desempenho do MLP depende
fortemente da escolha adequada de hiperparâmetros e do pré-processamento dos dados,
conforme discutido na literatura~\cite{haykin_neural_networks}.

