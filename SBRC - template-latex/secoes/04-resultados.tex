\section{Resultados e Discussão}

\subsection{Resultados com K-Nearest Neighbors}

Após alguns testes, os resultados mais precisos foram encontrados com $K=1$, resultando
em um F1-Score de 0.9946. Provavelmente, esse comportamento ocorre devido à grande
quantidade de dados disponíveis para cada entrada de treinamento, o que torna a maioria
dos ataques mais evidentes. Valores superiores de $K$ levaram ao sobreajuste e ao
decréscimo progressivo do desempenho do modelo.

Após uma análise baseada na remoção individual de cada atributo, observou-se que o
parâmetro \textit{hot} foi o mais decisivo para o modelo, reduzindo o F1-Score para
0.9931 quando removido. Outra informação relevante é que oito atributos, quando
removidos, aumentaram a precisão do modelo. Após essa remoção, o F1-Score atingiu
0.9960, sendo o atributo \textit{diff\_srv\_rate} o mais prejudicial, cuja exclusão
resultou em um aumento de 0.0005 pontos no F1-Score.

\begin{table}[ht]
\centering
\caption{Métricas de desempenho do KNN}
\label{tab:if_metricsKNN}
\begin{tabular}{lcc}
\hline
\textbf{Métrica} & \textbf{Classe Normal} & \textbf{Classe Anomalia} \\
\hline
Precision & 0.99 & 0.99 \\
Recall    & 0.99 & 0.99 \\
F1-Score  & 0.99 & 0.99 \\
Support   & 4034 & 3524 \\
\hline
\end{tabular}
\end{table}

\subsection{Resultados com Isolation Forest}

No entanto, o Isolation Forest se mostrou ineficiente quando comparado aos outros
modelos utilizados durante a pesquisa, apresentando a menor acurácia (0.748).
Entretanto, a acurácia não é a métrica mais confiável em conjuntos de dados
desbalanceados, o que é comum em problemas de detecção de anomalias, pois pode ser
inflacionada pela capacidade do modelo em classificar corretamente a classe
majoritária (tráfego normal). Dessa forma, uma análise mais aprofundada torna-se
necessária.

\begin{table}[ht]
\centering
\caption{Métricas de desempenho do Isolation Forest}
\label{tab:if_metricsIF}
\begin{tabular}{lcc}
\hline
\textbf{Métrica} & \textbf{Classe Normal} & \textbf{Classe Anomalia} \\
\hline
Precision & 1.00 & 0.65 \\
Recall    & 0.53 & 1.00 \\
F1-Score  & 0.69 & 0.79 \\
Support   & 2690 & 2349 \\
\hline
\end{tabular}
\end{table}

Como pode ser observado na Tabela~\ref{tab:if_metrics}, todos os eventos classificados
como normais eram, de fato, normais. A principal dificuldade do modelo esteve na
identificação correta das anomalias, apresentando precisão de 65\% (0.65) ao rotular
eventos como intrusão. Além disso, considerando todos os eventos analisados, o modelo
foi capaz de identificar corretamente apenas 53\% (0.53) dos eventos normais, o que
indica a geração de um elevado número de falsos positivos, isto é, eventos normais
classificados como ataques.

Por outro lado, o Isolation Forest não apresentou o problema de gerar falsos negativos,
uma vez que ataques não foram classificados como eventos normais. Em síntese, o modelo
apresenta dificuldades em verificar se um evento é realmente normal, gerando falsos
positivos que podem ocasionar retrabalho para equipes de segurança, as quais precisam
analisar manualmente eventos normais identificados como anômalos.

\subsection{Resultados com Random Forest}
CHANGE ME
% Texto a ser inserido

\subsection{Resultados com MLP}
CHANGE ME
% Texto a ser inserido

