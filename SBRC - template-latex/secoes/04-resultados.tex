\section{Resultados e Discussão}

\subsection{Resultados com K-Nearest Neighbors}

Para o modelo KNN, foram realizados alguns testes, com os resultados mais precisos foram encontrados com $K=1$, resultando
em um F1-Score de 0.9946 e uma acurácia de 0.9960. Provavelmente, esse comportamento ocorre devido à grande
quantidade de dados disponíveis para cada entrada de treinamento, o que torna a maioria
dos ataques mais evidentes. Valores superiores de $K$ levaram ao sobreajuste e ao
decréscimo progressivo do desempenho do modelo.

Após uma análise baseada na remoção individual de cada atributo, observou-se que o
parâmetro \textit{hot} foi o mais decisivo para o modelo, reduzindo o F1-Score para
0.9931 quando removido. Outra informação relevante é que oito atributos, quando
removidos, aumentaram a precisão do modelo. Após essa remoção, o F1-Score atingiu
0.9960, sendo o atributo \textit{diff\_srv\_rate} o mais prejudicial, cuja exclusão
resultou em um aumento de 0.0005 pontos no F1-Score.

\begin{table}[ht]
\centering
\caption{Métricas de desempenho do KNN}
\label{tab:knn_metrics}
\begin{tabular}{lcc}
\hline
\textbf{Métrica} & \textbf{Classe Normal} & \textbf{Classe Anomalia} \\
\hline
Precision & 0.99 & 0.99 \\
Recall    & 0.99 & 0.99 \\
F1-Score  & 0.99 & 0.99 \\
Support   & 4034 & 3524 \\
\hline
\end{tabular}
\end{table}

\subsection{Resultados com Isolation Forest}

O Isolation Forest se mostrou ineficiente quando comparado aos outros
modelos utilizados durante a pesquisa, apresentando a menor acurácia (0.748).
Entretanto, a acurácia não é a métrica mais confiável em conjuntos de dados
desbalanceados, o que é comum em problemas de detecção de anomalias, pois pode ser
inflacionada pela capacidade do modelo em classificar corretamente a classe
majoritária (tráfego normal). Dessa forma, uma análise mais aprofundada torna-se
necessária.

\begin{table}[ht]
\centering
\caption{Métricas de desempenho do Isolation Forest}
\label{tab:if_metrics}
\begin{tabular}{lcc}
\hline
\textbf{Métrica} & \textbf{Classe Normal} & \textbf{Classe Anomalia} \\
\hline
Precision & 1.00 & 0.65 \\
Recall    & 0.53 & 1.00 \\
F1-Score  & 0.69 & 0.79 \\
Support   & 2690 & 2349 \\
\hline
\end{tabular}
\end{table}

Como pode ser observado na Tabela~\ref{tab:if_metrics}, todos os eventos classificados
como normais eram, de fato, normais. A principal dificuldade do modelo esteve na
identificação correta das anomalias, apresentando precisão de 65\% (0.65) ao rotular
eventos como intrusão. Além disso, considerando todos os eventos analisados, o modelo
foi capaz de identificar corretamente apenas 53\% (0.53) dos eventos normais, o que
indica a geração de um elevado número de falsos positivos, isto é, eventos normais
classificados como ataques.

Por outro lado, o Isolation Forest não apresentou o problema de gerar falsos negativos,
uma vez que ataques não foram classificados como eventos normais. Em síntese, o modelo
apresenta dificuldades em verificar se um evento é realmente normal, gerando falsos
positivos que podem ocasionar retrabalho para equipes de segurança, as quais precisam
analisar manualmente eventos normais identificados como anômalos.

\subsection{Resultados com Random Forest}

O Random Forest apresentou o melhor desempenho entre todos os modelos avaliados quando
utilizado com o conjunto completo de atributos, alcançando uma acurácia de 0.9975. O
modelo demonstrou excelente capacidade de generalização, classificando corretamente
tanto o tráfego normal quanto as intrusões, o que evidencia seu elevado poder
discriminativo em um cenário de detecção de anomalias em redes.

Conforme apresentado na Tabela~\ref{tab:rf_metrics_full}, os valores de Precision,
Recall e F1-Score foram iguais a 1.00 para ambas as classes, indicando a ausência
praticamente total de falsos positivos e falsos negativos. Esses resultados reforçam a
robustez do Random Forest e justificam sua utilização como modelo de referência para a
análise de importância dos atributos. Nesse contexto, a feature \textit{src\_bytes} foi
identificada como a mais influente no modelo, motivando sua remoção em uma segunda etapa
experimental.

Com o objetivo de avaliar a robustez do Random Forest frente a conjuntos de dados
incompletos, o modelo foi reavaliado após a exclusão da feature \textit{src\_bytes}.
Nesse cenário, observou-se uma queda significativa no desempenho, com a acurácia
reduzida para 0.9630, conforme apresentado na Tabela~\ref{tab:rf_metrics_reduced}.
Embora o modelo ainda apresente bons resultados globais, nota-se uma redução no Recall
da classe normal (0.93), indicando um aumento na taxa de falsos positivos, ou seja,
eventos normais classificados incorretamente como anomalias.

Esse comportamento evidencia a forte dependência do Random Forest em relação a
atributos altamente informativos e reforça a importância da qualidade e da completude
dos dados em aplicações de detecção de intrusões em redes. Além disso, os resultados
obtidos corroboram a motivação deste trabalho em investigar como diferentes modelos de
aprendizado de máquina reagem à ausência de atributos relevantes no conjunto de dados.
\begin{table}[ht]
\centering
\caption{Métricas de desempenho do Random Forest (dataset completo)}
\label{tab:rf_metrics_full}
\begin{tabular}{lcc}
\hline
\textbf{Métrica} & \textbf{Classe Normal} & \textbf{Classe Anomalia} \\
\hline
Precision & 1.00 & 1.00 \\
Recall    & 1.00 & 1.00 \\
F1-Score  & 1.00 & 1.00 \\
Support   & 3523 & 4035 \\
\hline
\end{tabular}
\end{table}


\begin{table}[ht]
\centering
\caption{Métricas de desempenho do Random Forest (dataset reduzido, sem \textit{src\_bytes})}
\label{tab:rf_metrics_reduced}
\begin{tabular}{lcc}
\hline
\textbf{Métrica} & \textbf{Classe Normal} & \textbf{Classe Anomalia} \\
\hline
Precision & 0.99 & 0.94 \\
Recall    & 0.93 & 0.99 \\
F1-Score  & 0.96 & 0.97 \\
Support   & 3523 & 4035 \\
\hline
\end{tabular}
\end{table}


\subsection{Resultados com MLP}

O Multilayer Perceptron foi avaliado em dois cenários distintos: um utilizando todas as
features disponíveis no conjunto de dados e outro com a remoção da feature
\textit{src\_bytes}, identificada como a mais influente no modelo Random Forest, que
apresentou o melhor desempenho geral. O objetivo dessa análise foi avaliar o impacto da
ausência de uma feature relevante no desempenho do MLP, simulando cenários em que o
conjunto de dados pode estar incompleto.

No cenário completo, o MLP obteve uma acurácia de 0.9960, apresentando desempenho
equilibrado entre as classes. A classe Anomalia apresentou precisão de 0.9969 e recall
de 0.9946, enquanto a classe Normal obteve precisão de 0.9953 e recall de 0.9973,
resultando em F1-Scores superiores a 0.995 para ambas as classes, conforme apresentado
na Tabela~\ref{tab:mlp_full}.

\begin{table}[ht]
\centering
\caption{Métricas de desempenho do MLP com todas as features}
\label{tab:mlp_full}
\begin{tabular}{lcc}
\hline
\textbf{Métrica} & \textbf{Classe Anomalia} & \textbf{Classe Normal} \\
\hline
Precision & 0.9969 & 0.9953 \\
Recall    & 0.9946 & 0.9973 \\
F1-Score  & 0.9957 & 0.9963 \\
Support   & 3523   & 4035   \\
\hline
\end{tabular}
\end{table}

Após a remoção da feature \textit{src\_bytes}, o MLP apresentou uma leve redução de
desempenho, com acurácia de 0.9952. Observa-se que, embora o recall da classe Anomalia
tenha aumentado para 0.9969, houve redução na precisão dessa classe, passando para
0.9929. Para a classe Normal, verificou-se comportamento inverso, com aumento da
precisão e redução do recall. Ainda assim, os valores de F1-Score permaneceram elevados
para ambas as classes, indicando que o modelo manteve desempenho consistente mesmo com a
remoção de uma feature altamente relevante em outro classificador, conforme mostrado na
Tabela~\ref{tab:mlp_reduced}.

\begin{table}[ht]
\centering
\caption{Métricas de desempenho do MLP sem a feature \textit{src\_bytes}}
\label{tab:mlp_reduced}
\begin{tabular}{lcc}
\hline
\textbf{Métrica} & \textbf{Classe Anomalia} & \textbf{Classe Normal} \\
\hline
Precision & 0.9929 & 0.9973 \\
Recall    & 0.9969 & 0.9938 \\
F1-Score  & 0.9949 & 0.9955 \\
Support   & 3523   & 4035   \\
\hline
\end{tabular}
\end{table}

De forma geral, os resultados indicam que o MLP é relativamente robusto à remoção de uma
feature altamente influente em outro modelo, apresentando apenas pequenas variações nas
métricas de desempenho. Esse comportamento reforça a capacidade do MLP de aprender
representações distribuídas dos dados, reduzindo sua dependência de atributos
individuais.

