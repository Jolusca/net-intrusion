
\section{Dataset}

O conjunto de dados utilizado foi obtido a partir da plataforma Kaggle, no repositório
\textit{Network Intrusion Detection}, disponibilizado publicamente por
Sampada Bhosale~\cite{dataset_kaggle}.

O dataset foi construído a partir da simulação de um ambiente de rede militar típico da
Força Aérea dos Estados Unidos, desenvolvido para representar diferentes tipos de tráfego
de rede, incluindo tanto comunicações legítimas quanto diversos tipos de ataques.
Cada instância do conjunto de dados representa uma conexão de rede, contendo atributos
como a quantidade de bytes transmitidos pela origem, a duração da conexão e outros
parâmetros, conforme descrito na tabela apresentada no Apêndice.

O conjunto de dados contém aproximadamente 25 mil instâncias, descritas por 42
atributos. Após a simulação supervisionada do ambiente, as conexões foram rotuladas como
tráfego normal ou intrusão, caracterizando um problema de classificação binária, no qual
a variável alvo assume os valores \textit{Normal} ou \textit{Anomalia}. A descrição
detalhada dos atributos utilizados é apresentada no Apêndice deste trabalho.
\section{Pré-processamento}
\subsection{Tratamento de Dados}
Devido às diferenças entre os modelos avaliados, as etapas de pré-processamento
foram adaptadas aos requisitos específicos de cada classificador, mantendo a
comparabilidade dos resultados. Inicialmente, o conjunto de dados foi separado
em atributos preditores e rótulos, seguido da divisão em conjuntos de treinamento
e teste por meio de amostragem estratificada.

As variáveis categóricas foram codificadas por meio de \textit{One-Hot Encoding}
ou \textit{Label Encoding}, conforme o modelo utilizado, enquanto as variáveis
numéricas foram padronizadas quando necessário. Além disso, foram conduzidos
experimentos com conjuntos de dados reduzidos, nos quais atributos relevantes
foram removidos de forma controlada, com o objetivo de avaliar a robustez dos
modelos frente à ausência parcial de informações.


\subsection{Seleção e remoção de Atributos}
A aplicação da Análise de Componentes Principais (PCA) como etapa de pré-processamento
tem se mostrado eficaz para a redução da dimensionalidade dos dados sem perda
significativa de informação. Estudos indicam que o PCA é capaz de preservar cerca de
99\% da variância original mesmo com reduções superiores a 50\% no número de atributos,
conforme observado por Santos e Miani~\cite{santos_reducao_dimensao_intrusao}. Nesse
sentido, técnicas de redução e seleção de atributos são amplamente empregadas na
literatura como forma de mitigar redundâncias e melhorar a generalização dos modelos.

Neste artigo, entretanto, optou-se por não aplicar técnicas automáticas de redução de
dimensionalidade, como o PCA. Em vez disso, foram realizadas diferentes análises
utilizando modelos de aprendizagem de máquina com o objetivo de identificar as
\textit{features} de maior importância para cada classificador. Em seguida, buscando
simular um cenário mais próximo de um ambiente real, no qual nem sempre todos os
atributos estão disponíveis para os algoritmos de análise, foram conduzidos testes
com a remoção manual das \textit{features} mais relevantes. Essa abordagem permitiu
analisar a influência individual de cada atributo no desempenho dos modelos, bem como
avaliar a robustez de cada método frente à ausência parcial de informações.




