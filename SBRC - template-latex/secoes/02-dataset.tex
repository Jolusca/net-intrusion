\section{Dataset e pré-processamento}
\subsection{Dataset}

O conjunto de dados utilizado foi obtido a partir da plataforma Kaggle, no repositório
\textit{Network Intrusion Detection}, disponibilizado publicamente por
Sampada Bhosale~\cite{dataset_kaggle}.

O dataset foi construído a partir da simulação de um ambiente de rede militar típico da
Força Aérea dos Estados Unidos, desenvolvido para representar diferentes tipos de tráfego
de rede, incluindo tanto comunicações legítimas quanto diversos tipos de ataques.
Cada instância do conjunto de dados representa uma conexão de rede, contendo atributos
como a quantidade de bytes transmitidos pela origem, a duração da conexão e outros
parâmetros, conforme descrito na tabela apresentada no Apêndice.

Por fim, após as simulações realizadas de forma supervisionada, as conexões foram
classificadas como tráfego normal ou como intrusão, fazendo com que a variável alvo
seja do tipo binária, assumindo os valores \textit{Normal} ou \textit{Anomalia}.

\subsection{Pré-processamento}

A aplicação da Análise de Componentes Principais (PCA) como etapa de pré-processamento
tem se mostrado eficaz para a redução da dimensionalidade dos dados sem perda
significativa de informação. Estudos indicam que o PCA é capaz de preservar cerca de
99\% da variância original mesmo com reduções superiores a 50\% no número de atributos,
conforme observado por Santos e Miani~\cite{santos_reducao_dimensao_intrusao}.

Neste artigo, foram realizadas diferentes análises utilizando modelos de aprendizagem
de máquina, com o objetivo de identificar as \textit{features} de maior importância
para cada modelo. Em seguida, buscando simular um cenário mais próximo de um ambiente
real, no qual nem sempre todos os atributos estão disponíveis para os algoritmos de
análise, foram conduzidos testes com a remoção das \textit{features} mais relevantes.
Essa abordagem teve como objetivo analisar a influência de cada atributo no desempenho
dos modelos, bem como avaliar a robustez de cada método frente à ausência parcial de
informações.



