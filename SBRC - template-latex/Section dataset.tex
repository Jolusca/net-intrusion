\section{Dataset e pré-processamento}
Para este artigo, foi utilizado um dataset obtido na plataforma Kaggle, consistindo em um conjunto de dados no formato .CSV voltado à detecção de intrusão em redes de computadores.

O dataset foi construído a partir da simulação de um ambiente de rede militar típico da Força Aérea dos Estados Unidos, desenvolvido para representar diferentes tipos de tráfego de rede, incluindo tanto comunicações legítimas quanto diversos tipos de ataques.

Cada instância do conjunto de dados representa uma conexão de rede, contendo atributos como a quantidade de bytes transmitidos pela origem, a duração da conexão, entre outros parâmetros, conforme descrito na Tabela a seguir.

\begin{table}[htbp]
\caption{Descrição dos principais atributos do dataset}
\label{tab:dataset_features}
\centering
\begin{tabular}{|p{3cm}|p{5cm}|}
\hline
\textbf{Atributo} & \textbf{Descrição} \\ \hline
duration & Duração da conexão de rede \\ \hline
protocol\_type & Protocolo utilizado na conexão (e.g., TCP, UDP, ICMP) \\ \hline
service & Serviço de rede acessado pela conexão \\ \hline
flag & Estado da conexão conforme definido pelo protocolo \\ \hline
src\_bytes & Quantidade de bytes enviados pela origem \\ \hline
dst\_bytes & Quantidade de bytes recebidos pelo destino \\ \hline
land & Indica se a conexão é local (origem e destino iguais) \\ \hline
wrong\_fragment & Número de fragmentos incorretos \\ \hline
urgent & Número de pacotes urgentes \\ \hline
num\_failed\_logins & Tentativas de login malsucedidas \\ \hline
logged\_in & Indica se o login foi realizado com sucesso \\ \hline
num\_compromised & Número de condições comprometidas \\ \hline
root\_shell & Indica a obtenção de shell com privilégio root \\ \hline
count & Número de conexões para o mesmo host em curto intervalo \\ \hline
srv\_count & Número de conexões para o mesmo serviço \\ \hline
class & Rótulo da conexão (normal ou ataque) \\ \hline
\end{tabular}
\end{table}


Por fim, após as simulações realizadas de forma supervisionada, as conexões foram classificadas como tráfego normal ou como intrusão, fazendo com que a variável alvo seja do tipo binária, assumindo os valores Normal ou Anomalia.